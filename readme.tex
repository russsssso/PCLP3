\documentclass[11pt, a4paper]{article}
\usepackage[utf8]{inputenc}
\usepackage[T1]{fontenc}
\usepackage{amsmath}
\usepackage{amsfonts}
\usepackage{amssymb}
\usepackage{graphicx}
\usepackage{geometry}
\usepackage{hyperref}
\usepackage{listings} 
\usepackage{array}
\usepackage{longtable} 
\usepackage{float} 
\usepackage{enumitem} 
\usepackage{titling}

\geometry{a4paper, margin=1in}

\hypersetup{
    colorlinks=true,
    linkcolor=blue,
    filecolor=magenta,
    urlcolor=cyan,
    pdftitle={README - Proiect PCLP3 Partea I},
    pdfpagemode=FullScreen,
}

\lstdefinestyle{mystyle}{
    backgroundcolor=\color{lightgray!20},
    commentstyle=\color{green!50!black},
    keywordstyle=\color{blue},
    numberstyle=\tiny\color{gray},
    stringstyle=\color{purple},
    basicstyle=\ttfamily\footnotesize,
    breakatwhitespace=false,
    breaklines=true,
    captionpos=b,
    keepspaces=true,
    numbers=left,
    numbersep=5pt,
    showspaces=false,
    showstringspaces=false,
    showtabs=false,
    tabsize=2
}
\lstset{style=mystyle}

\newcommand{\studentname}{Rusu Andrei Ionut C2 341}
\title{Proiect PCLP3 - Partea I }
\date{\today}

% Redefine \maketitle to include student info at the top
\makeatletter
\renewcommand{\maketitle}{
  \begin{center}
    \Large \studentname \\
    \Huge \@title \\[1.5em]
    \large \@date
    \vspace{2em}
  \end{center}
  \par
  \thispagestyle{empty}
}
\makeatother

\begin{document}
\maketitle
\thispagestyle{empty}
\clearpage
\pagenumbering{arabic}

\section{Setul de Date}

Pentru aceasta tema am ales setul de date \href{https://www.kaggle.com/datasets/efeyldz/plant-communication-dataset-classification?resource=download}{Plant Communication Dataset} (pare sa fie generat sintetic) de pe platforma Kaggle, pentru o problema de clasificare.

Setul de date initial a fost augmentat prin adaugarea a trei noi coloane:
\sloppy
\begin{itemize}
    \item  \textbf{Soil\_Nutrient\_Level}: O coloana pentru nivelul de nutrienti din sol. Valorile au fost generate dintr-o distributie normala, cu medii si deviatii standard specifice fiecarui tip de mesaj (\texttt{Plant\_Message\_Type}), pentru a reflecta o legatura intre starea plantei si nutrienti - am folosit deviatii standard mari a.i. plajele de valori sa se intercaleze mult, pentru a nu crea o coloana puternic corelata cu coloana target.
    \item \textbf{Temperature\_Stress\_Factor}: O coloana categorica ('Low', 'Medium', 'High') derivata din \texttt{Ambient\_Temperature\_C}, indicand nivelul de stres termic.
    \item \textbf{Photosynthetic\_Efficiency\_Index}: Un indice calculat pe baza factorilor de stres termic, umiditate, nutrienti si expunere la soare, reflectand eficienta fotosintetica.
\end{itemize}

\subsection{Introducerea Zgomotului}
S-a adaugat zgomot la coloanele numerice relevante. Acest zgomot a fost generat dintr-o distributie normala cu media 0 si o deviatie standard egala cu 2\% din deviatia standard a fiecarei coloane. Coloanele afectate includ cele initiale numerice si cele nou create (\texttt{Soil\_Nutrient\_Level}, \  \texttt{Photosynthetic\_Efficiency\_Index}). Valorile au fost ulterior limitate pentru a ramane in intervale plauzibile.

\subsection{Simularea Valorilor Lipsa (NaN)}
\sloppy
S-au introdus valori NaN in mod aleatoriu pentru 5\% din instante in urmatoarele coloane: \texttt{Pollen\_Scent\_Complexity},\ \texttt{Bioluminescence\_Intensity\_Lux},\ \texttt{Growth\_Rate\_mm\_day},\ \texttt{Soil\_Nutrient\_Level},\ si \texttt{Soil\_Moisture\_Level}.



\end{document}